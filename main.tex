\documentclass{article}

\usepackage[utf8]{inputenc}
\usepackage{parskip}

\usepackage{amssymb,amsmath,amsthm}
\usepackage{mathtools} 
\usepackage{bbm}

\newtheorem{theorem}{Theorem}[section]
\newtheorem*{theorem*}{Theorem}
\theoremstyle{definition}
\newtheorem{definition}{Definition}[section]

\newcommand{\R}{\mathbb{R}}
\newcommand{\N}{\mathbb{N}}
\newcommand{\X}{\mathcal{X}}
\newcommand{\F}{\mathcal{F}}
\newcommand{\Y}{\mathcal{Y}}
\newcommand{\HH}{\mathcal{H}}
\newcommand{\D}{\mathcal{D}}

\newcommand{\set}[1]{\left{#1\right}}
\newcommand{\xsample}[1]{x_1,\ldots,x_{#1}}
\newcommand{\xtagged}[1]{(x_1,y_1),\ldots,(x_{#1},y_{#1})}
\newcommand{\sample}{\xsample{m}}
\newcommand{\tagged}{\xtagged{m}}

\newcommand{\bigO}[1]{\mathcal{O}\left(#1\right)}
\newcommand{\uf}[1]{\frac{1}{#1}}
\DeclarePairedDelimiter\abs{\lvert}{\rvert}

\title{Compression - Recent Results}
\author{Meni Sadigurschi}
\date{August 2018}

\begin{document}

\maketitle

\section{Agnostic Compression}
\subsection{Definitions}

Let $S = \tagged$ be a sample drawn i.i.d from some unknown distribution $\D$. 
Let $(\kappa,\rho)$  be a selection scheme
and denote by $f_S := \rho(\kappa(S))$.

David et. al \cite{david2016supervised} defined the following definition -

\begin{definition}[Agnostic Sample Compression]
A selection scheme $(\kappa,\rho)$ 
is an \emph{agnostic sample compression scheme} 
for $\HH$ if for every sample $S$
\[ 
\uf{m}\sum_{i=1}^{m} \abs{f_S (x_i) - y_i} \leq 
\inf_{h\in\HH} \uf{m}\sum_{i=1}^{m} \abs{h (x_i) - y_i}
\] 
\end{definition}

\begin{definition}[$\epsilon$-Approximate Agnostic Compression Scheme] \label{approx-agnos-def}
A selection scheme $(\kappa,\rho)$ 
is an \emph{$\epsilon$-Approximate Agnostic Sample Compression Scheme} for $\HH$ 
if for every sample $S$
\[ 
\uf{m}\sum_{i=1}^{m} \abs{f_S (x_i) - y_i} \leq 
\inf_{h\in\HH} \uf{m}\sum_{i=1}^{m} \abs{h (x_i) - y_i} + \epsilon
\] 
\end{definition}

We defined the more strict, and more natural, definition -

\begin{definition}[Uniformly $\epsilon$-Approximate Agnostic Compression Scheme]
A selection scheme $(\kappa,\rho)$ 
is an \emph{Uniformly $\epsilon$-Approximate Agnostic Compression Scheme} for $\HH$ 
if for every sample $S$
\[ \max_{i\in [m} \abs{f_S (x_i) - y_i} \leq \epsilon \]
\end{definition}

\subsection{Prior Results}

At section 4.1 on the above paper they prove the following theorem

\begin{theorem*}[Theorem 4.1]
There is no agnostic sample compression scheme for zero-dimensional linear regression with size $\leq m/2$.
\end{theorem*}

This impressibility was the motivation to define \ref{approx-agnos-def}. Then they demonstrated that the zero-regression problem have an $\epsilon$-Approximate Agnostic Compression Scheme of size $\bigO{\log(1/\epsilon)/\epsilon}$, by choosing randomly a sub-sample of size $\lceil 1/\epsilon \rceil$ 
and adding $\bigO{\log(1/\epsilon)/\epsilon}$ bits to encode the sub-sample.

In addition they proved (Theorem 4.2 and 4.3) a connection between $\epsilon$-Approximate Agnostic Compression Scheme and agnostic-learnability.

\subsection{Ours}

With our definition in mind we look at the somewhat similar problem of zero-regression, in other words, compressing the sample so one can approximately recover its mean.
we prove the following Theorem

\begin{theorem} \label{no-uniform}
There is no Uniformly $\epsilon$-Approximate Agnostic Compression Scheme for zero-dimensional linear regression with size 
$ \bigO{\log(\epsilon)} $.
\end{theorem}

We note that in order to achieve Uniformly $ \bigO{\log(\epsilon)} $-Approximate Agnostic Compression 
one can just take as side-information the representation of the approximate mean  (as a floating point).
In that light \ref{no-uniform} meaning is that no non-trivial Uniformly approximate compression scheme exist for this problem.

\section{Covering numbers}

\section{Future Directions}

\bibliographystyle{unsrt}
\bibliography{general}

\end{document}
